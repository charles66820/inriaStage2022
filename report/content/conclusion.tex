\section{Conclusion}

Le premier objectif est en partie rempli, car j'ai implémenté une partie des opérations et
j'ai bien vu les différences entre la manipulation des vecteurs dans \Eigen et \MIPP.
Par-contre je n'ai pas pu faire une campagne d'évaluation des performances, ni le test
d'\Eigen sur l'architecture \emph{Risc-V}, car toutes les opérations ne sont pas codées en
\emph{code MIPP}.

\subsection{Ce que j'ai remarqué}

\begin{itemize}
  \item Les tests unitaires d'\Eigen sont très longs ce qui fait qu'ils ne m'ont pas été utiles
  pour tester mes modifications.
  \item Les tests unitaires d'\Eigen ont des bugs ce qui fait qu'il faut les lancer
  plusieurs fois pour qu'ils fonctionnent.
  \item J'ai remarqué que plus je codais en \emph{code MIPP} dans \Eigen, plus le temps de
  compilation était élevé. C'est très certainement dû aux templates mais je ne peux pas dire
  si cela vient des templates dans \MIPP ou dans \Eigen ou les deux.
\end{itemize}

Grâce à mes tests et mes implémentations \MIPP dans \Eigen j'ai pu lister des opérations
qui manquent dans \MIPP :

\begin{table}[H]
  \centering
  \caption*{\texttimes: absent de \MIPP \checkmark: présent dans \MIPP}
  \begin{tabularx}{\linewidth}[H]{|m{.238\linewidth}|m{.1205\linewidth}|m{.0872\linewidth}|m{.0705\linewidth}|m{.1594\linewidth}|X|} %m{.108\linewidth}
  % \begin{tabularx}{\linewidth}[H]{|l|c|c|c|c|c|}
    \hline
                     & \textbf{AVX512} & \textbf{AVX2} & \textbf{AVX} & \textbf{SSE4.1/4.2} & \textbf{SSE2/3} \\
    \hline
    add<int32_t>    & \checkmark      & \checkmark    & \texttimes   & \checkmark          & \checkmark      \\
    \hline
    sub<int32_t>    & \checkmark      & \checkmark    & \texttimes   & \checkmark          & \checkmark      \\
    \hline
    mul<int32_t>    & \checkmark      & \checkmark    & \texttimes   & \checkmark          & \texttimes      \\
    \hline
    orb<int8_t>     & \checkmark      & \checkmark    & \texttimes   & \checkmark          & \checkmark      \\
    \hline
    xor<int8_t>     & \checkmark      & \checkmark    & \texttimes   & \checkmark          & \checkmark      \\
    \hline
    and<int8_t>     & \checkmark      & \checkmark    & \texttimes   & \checkmark          & \checkmark      \\
    \hline
    cmpneq<int16_t> & \texttimes      & \checkmark    & \texttimes   & \texttimes          & \texttimes      \\
    \hline
    cmpneq<int8_t>  & \texttimes      & \checkmark    & \texttimes   & \texttimes          & \texttimes      \\
    \hline
  \end{tabularx}
  \caption{Abstractions \MIPP non implémentées}
\end{table}

Pour exécuter le code en \emph{AVX512} j'ai utilisé \emph{PlaFRIM}.

\subsection{Les connaissances que j'ai acquises lors de mes études et que j'ai utilisées au
cours du stage}

\begin{itemize}
  \item Les tests de non-régression que j'ai dû implémenter pour vérifier que mes
  implémentations étaient conformes à la version précédente. J'ai obtenu ces compétences
  lors des cours d'Architecture Logiciel (AL), de Projet de Programmation (PdP) et dans
  un cours de BTS (SLAM4).
  \item La factorisation de code et l'utilisation des 5 principes solides que j'ai
  appris en cours d'Architecture Logiciel (AL) et de Projet de Programmation (PdP).
  \item L'utilisation des intrasics et le calcul vectoriel que j'ai vu en Programmation
  sur Architecture Parallèles (PAP).
  \item Les bases en C++ que j'ai vu un petit peu dans Nachos en TP de Système
  d'Exploitation (SE) mais aussi dans le projet PdP de mon équipe.
  \item La compréhension du monde de la recherche grâce à l'UE Initiation recherche de L3.
\end{itemize}

\subsection{Ce que m'a apporté le stage}

\begin{itemize}
  \item Une meilleure compréhension des casts, des conversions des types de base et de la
  manipulation des vecteurs.
  \item Une nouvelle expérience de développement dans un projet open source qui est très
  utilisé.
  \item L'utilisation avancée des templates en C++.
  \item Des utilisations différentes des vecteurs avec plusieurs visions de leur
  utilisation.
  \item Une meilleure compréhension du monde de la recherche.
\end{itemize}
